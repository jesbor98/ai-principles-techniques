% The format (A4, 10pt, one sided) should NOT be changed. 
\documentclass[a4paper,10pt,oneside]{article}

% The package babel is loaded set up for Swedish with Swedish 
% hyphenation,replaces "Contents" with "Innehållsförteckning, 
% "References" with "Litteraturförteckning", etc.
\usepackage[swedish]{babel}

\usepackage[T1]{fontenc}

% The package "inputenc" lets us specify what character encoding
% has been used to save the .tex file. Make sure you set it up
% with the right character encoding, otherwise ÅÄÖ might look 
% wrong, or possibly the document won't compile at all.
\usepackage[utf8]{inputenc}     % Most likely right nowadays, 
                                % might even be standard and not necessary
% \usepackage[latin1]{inputenc} % Possibly right if you use Windows
% Other alternatives are available, but much less likely to be used

% The packages listed below are optional and can be removed if you
% don't use them 
\usepackage{graphicx} 
\usepackage{cite}
\usepackage{url}
\usepackage{ifthen}
\usepackage{listings}	

% These two lines set up options for the listings package and
% can be removed if you don't use it, or changed if you, e.g, 
% use another language than Java. 
% For more information about the listings package see:
% ftp://ftp.tex.ac.uk/tex-archive/macros/latex/contrib/listings/listings.pdf
\def \lstlistingname {Kodexempel}
\lstset{language=Java,tabsize=3,numbers=left,frame=L,floatplacement=hbtp}


\usepackage{ifpdf}
\ifpdf
	\usepackage[hidelinks]{hyperref}
\else
	\usepackage{url}
\fi

% Ändra inte på titeln
\title{Programming report exercise}

% Write the name and user namn for all participants in the group here.
% Separate persons with \and
\author{Amanda Enhörning \url{s1128126} \and Jessica Borg \url{s1129470}}

\begin{document}

% Do NOT change the title format in any way, especially not to place it on 
% a separate page. Rememeber that you have a *MAXIMUM* of two pages, including
% title...
\maketitle

\section{Introduction}

Sketch the context of the work so that the reader understand what they are reading. In this
case: write that you will empirically compare two sorting algorithms on a set of inputs to get
an idea on the difference between worst-case and best-case running times for both algorithms.
This is quite similar as the Introduction in the example report.

\section{Specification}

Specify the task at hand, with the assignment text as your guideline. Note that the algorithms
are already given in this example exercise, but normally you would describe what the software
should do (in this case: read in a comma-separated values (.csv) file with arrays, one per row
of the file, sort the array using Selection Sort and Insertion Sort, count the array assignments
and comparisons, and return the sorted array as well as the measures). This specification is
in the first part of the Methods in the example report.

\section{Design}

Specify here your software design. Again, the algorithm is already given, but here you would
specify the different classes and data structures you use, packages used for I/O and visuali-
sation etc., and the pseudo code for existing algorithms that you will implement. In this case
you would specify that you separate the sorting from the I/O (reading in the .csv file, writing
down the results), you would give the pseudo-code for the sorting algorithms, etc. The middle
part of the Methods in the example report contains this information.

\section{Implementation}

Give (only) the crucial part of the implementation here, preferably with line numbers, and
explain what it does without literally repeating the code. Highlight important aspects and
don’t repeat trivial translations from the pseudo-code. This is in the last part of the Methods
in the example report.

\section{Testing}

Explain how you will test and experiment on the implementations in order to get a good
idea of 1) their correct working and 2) the empirical results you are interested in. In this
case you would describe the contents of the .csv file (extremes, such as already sorted arrays,
arrays sorted in opposite order, uniform arrays; as well as several randomly sorted arrays) and
how you would interpret the results. Oftentimes the assignment will ask you for particular
empirical questions that you can summarize here.

\section{Results}

Write up the results in a clear and easy to follow way. Use a table and if meaningful, a
graph, to summarize results. Give the raw facts here, not the interpretation of the facts. For
example, a table like this could be useful:\\
\begin{tabular}{c|c|c|c|c}
	& \multicolumn{2}{c|}{Insertion} & \multicolumn{2}{c}{Selection} \\
	& Assignments & Comparisons & Assignments & Comparisons \\
	\hline
	Already sorted & \ldots & \ldots & \ldots & \ldots \\
	\ldots & \ldots & \ldots & \ldots & \ldots \\
	\end{tabular}

\section{Discussion}

In the discussion you interpret the results. What do you see and what does it mean (spoiler:
Selection Sort has the same number of comparisons and assignments for every n-sized array,
independent of the order, whereas Insertion Sort is sensitive to the order). The example report
could still be improved somewhat by more clearly separating raw results (running times) and
their interpretation.

\section{Conclusion and Reflection}

Conclude the report. Summarize the main findings, mirroring the introduction (in this case:
Selection Sort has the same worst-case and best-case run-time complexity, whereas Insertion
Sort is linear time in the best case). Reflect on the assignment: what obstacles did you
encounter? Any interesting observations? What could still be improved?
From this year on it is compulsory to give an overview of the work division. Any specific
information can go here (e.g., building on existing partial implementation of one of the team
members, someone dropping out or joining late, etc.). As a general rule: all team members
should contribute to both coding and writing the report!

\end{document}
